% Options for packages loaded elsewhere
\PassOptionsToPackage{unicode}{hyperref}
\PassOptionsToPackage{hyphens}{url}
\documentclass[
]{article}
\usepackage{xcolor}
\usepackage[margin=1in]{geometry}
\usepackage{amsmath,amssymb}
\setcounter{secnumdepth}{5}
\usepackage{iftex}
\ifPDFTeX
  \usepackage[T1]{fontenc}
  \usepackage[utf8]{inputenc}
  \usepackage{textcomp} % provide euro and other symbols
\else % if luatex or xetex
  \usepackage{unicode-math} % this also loads fontspec
  \defaultfontfeatures{Scale=MatchLowercase}
  \defaultfontfeatures[\rmfamily]{Ligatures=TeX,Scale=1}
\fi
\usepackage{lmodern}
\ifPDFTeX\else
  % xetex/luatex font selection
\fi
% Use upquote if available, for straight quotes in verbatim environments
\IfFileExists{upquote.sty}{\usepackage{upquote}}{}
\IfFileExists{microtype.sty}{% use microtype if available
  \usepackage[]{microtype}
  \UseMicrotypeSet[protrusion]{basicmath} % disable protrusion for tt fonts
}{}
\makeatletter
\@ifundefined{KOMAClassName}{% if non-KOMA class
  \IfFileExists{parskip.sty}{%
    \usepackage{parskip}
  }{% else
    \setlength{\parindent}{0pt}
    \setlength{\parskip}{6pt plus 2pt minus 1pt}}
}{% if KOMA class
  \KOMAoptions{parskip=half}}
\makeatother
\usepackage{graphicx}
\makeatletter
\newsavebox\pandoc@box
\newcommand*\pandocbounded[1]{% scales image to fit in text height/width
  \sbox\pandoc@box{#1}%
  \Gscale@div\@tempa{\textheight}{\dimexpr\ht\pandoc@box+\dp\pandoc@box\relax}%
  \Gscale@div\@tempb{\linewidth}{\wd\pandoc@box}%
  \ifdim\@tempb\p@<\@tempa\p@\let\@tempa\@tempb\fi% select the smaller of both
  \ifdim\@tempa\p@<\p@\scalebox{\@tempa}{\usebox\pandoc@box}%
  \else\usebox{\pandoc@box}%
  \fi%
}
% Set default figure placement to htbp
\def\fps@figure{htbp}
\makeatother
\setlength{\emergencystretch}{3em} % prevent overfull lines
\providecommand{\tightlist}{%
  \setlength{\itemsep}{0pt}\setlength{\parskip}{0pt}}
\usepackage{bookmark}
\IfFileExists{xurl.sty}{\usepackage{xurl}}{} % add URL line breaks if available
\urlstyle{same}
\hypersetup{
  pdftitle={Отчёт по лабораторной работе №1},
  pdfauthor={Пустобаев Леонид},
  hidelinks,
  pdfcreator={LaTeX via pandoc}}

\title{Отчёт по лабораторной работе №1}
\author{Пустобаев Леонид}
\date{2025-09-15}

\begin{document}
\maketitle

{
\setcounter{tocdepth}{2}
\tableofcontents
}
Цель работы Целью данной работы является изучение идеологии и применения
средств контроля версий, а также освоение практических навыков работы с
системой git.

Ход работы Базовая настройка git Я начал с настройки основных параметров
git, указав свое имя и email. Выполнил следующие команды:

git config --global user.name ``Иванов Иван'' git config --global
user.email
``\href{mailto:ivanov@example.com}{\nolinkurl{ivanov@example.com}}'' git
config --global core.quotepath false git config --global
init.defaultBranch master git config --global core.autocrlf input git
config --global core.safecrlf warn

Создание SSH-ключей Сгенерировал SSH-ключи для безопасного подключения к
удаленным репозиториям. Выполнил команды:

ssh-keygen -t rsa -b 4096 ssh-keygen -t ed25519

Создание PGP-ключей Создал PGP-ключ для подписи коммитов с помощью
команды:

gpg --full-generate-key

Выбрал тип RSA and RSA с размером 4096 бит и указал свои данные при
запросе.

Настройка GitHub Зарегистрировался на GitHub и добавил SSH-ключ в
настройках аккаунта. Для этого скопировал публичный ключ командой:

cat \textasciitilde/.ssh/id\_ed25519.pub

и добавил его в разделе SSH and GPG keys настроек GitHub.

Экспортировал PGP-ключ и также добавил его в настройках GitHub с помощью
команды:

gpg --armor --export и указал отпечаток ключа

Настройка автоматической подписи коммитов Настроил git для
автоматической подписи коммитов командами:

git config --global user.signingkey указал отпечаток PGP ключа git
config --global commit.gpgsign true

Создание рабочего пространства Создал каталог для лабораторных работ и
инициализировал git-репозиторий:

mkdir -p \textasciitilde/work/study/2023-2024/os-intro cd
\textasciitilde/work/study/2023-2024/os-intro git init

Создал тестовый файл и сделал первый коммит:

echo ``Начало работы над лабораторными'' \textgreater{} README.md git
add README.md git commit -S -m ``feat: initial commit''

Работа с удаленным репозиторием Связал локальный репозиторий с удаленным
на GitHub:

git remote add origin указал ссылку на репозиторий git push -u origin
master

Выводы В ходе выполнения лабораторной работы я освоил основные принципы
работы с системой контроля версий Git. Я научился настраивать рабочее
окружение, создавать и настраивать SSH и PGP ключи для безопасной
работы, а также подписывать коммиты для обеспечения их подлинности. Были
получены практические навыки работы с базовыми командами git: создание
репозитория, добавление файлов, создание коммитов и синхронизация с
удаленным репозиторием на GitHub. Все цели работы были успешно
достигнуты, и теперь я готов использовать git для последующих
лабораторных работ и проектов.

\end{document}
